\documentclass[a4paper,12pt]{article}
\usepackage[slovene]{babel}
\usepackage[utf8]{inputenc}
\usepackage[T1]{fontenc}
\usepackage{lmodern}
\usepackage{amsmath}
\usepackage{amsfonts}
\usepackage{amssymb}
\usepackage{enumitem}
\usepackage{fancyhdr}
\usepackage{hyperref}  
\usepackage{amsthm} 

\newtheorem{definicija}{Definicija}

\begin{document}
	\thispagestyle{empty}
	{\large
		\noindent Univerza v Ljubljani\\[1mm]
		Fakulteta za matematiko in fiziko\\[5mm]}
	\vfill
	
	\begin{center}{\Large
			{\bf Predstavitev 4-ciklov v snarkih}\\[2mm]
			Domen Skornšek, Žan Luka Kolarič\\[10mm]
			Mentorja: Janoš Vidali, Riste Škrekovski\\[2mm]
			Predmet: Finančna matematika \\[2mm]}
	\end{center}
	\vfill
	
	{\large
		Ljubljana, 2024}
	\pagebreak
	
	\section{Navodilo naloge}
	Želimo preveriti, ali (in kdaj) uvedba 4-ciklov v snark ohranja kromatični
	indeks(tj. kromatični indeks ostane 4). Uvedbo 4-cikla lahko izvedemo
	vsaj na dva načina:
	\begin{enumerate}
		\item Vzemite dva robova ab in cd v G in ju dvakrat razdelite, tako da dobite
		pot au1u2b iz roba ab in pot cv1v2d iz roba cd. Nato povežite u1 z v1
		in u2 z v2.
		\item Naj bo ab rob v G in a1, a2 druga dva soseda a, in naj bosta b1, b2
		druga dva soseda b. Zdaj odstranimo vrhova a, b in povežemo a1 z b1
		in a2 z b2.
	\end{enumerate}
	
	\section{Definicije}
	\begin{definicija}
		Če so vsa vozlišča grafa G enake stopnje
		k, pravimo, da je graf \textbf{k-regularen}; 3-regularnim grafom pravimo tudi \textbf{kubični grafi}.
	\end{definicija}
	
	\begin{definicija}
		Naj bo $k \in \mathbb{N}$ in $c': E(G) \to \{1, 2, \dots, k\}$ preslikava. Preslikavi $c'$ pravimo \textit{$k$-barvanje povezav} multigrafa brez zank $G$. Pravimo, da je barvanje povezav $c'$ \textit{pravilno}, če so povezave, ki imajo kako skupno krajišče, paroma različno obarvane. Najmanjše število $k$, za katerega obstaja pravilno $k$-barvanje povezav multigrafa brez zank, imenujemo \textit{kromatični indeks} multigrafa $G$ in ga označimo s $\chi'(G)$.
	\end{definicija}
	
	\begin{definicija}
		Dolžini najkrajšega cikla v grafu G pravimo ožina grafa G. Označimo jo z g(G).
	\end{definicija}
	\begin{definicija}
		Pravimo, da je A cikličen prerez, če ima G \textbackslash A dve komponenti in vsaka izmed teh dveh komponent vsebuje cikel.
	\end{definicija}
	
	\begin{definicija}
		Velikosti najmanjšega cikličnega prereza G je ciklična (povezavna) povezanost grafa G. Označimo jo z $\lambda_c(G)$.
	\end{definicija}
	
	\begin{definicija}
		Graf G je ciklično k-povezan po povezavah, če je $\lambda_c(G)$ $\geq$ k za
		naravno število k. Z drugimi besedami: odstraniti moramo prerez velikosti vsaj k,
		da G razpade na dve komponenti, kjer vsaka vsebuje cikel.
	\end{definicija}
	
	\begin{definicija}
		Snark je ciklično 4-povezan kubični graf z ožino n $\geq$ 5 in kromatičnim indeksom 4.
	\end{definicija}
	
	\section{Metoda dela}
	Odločila sva se, da bova uporabila prvi pristop torej, da bova vzela dva robova ab in cd v grafu G in ju dvakrat razdelila, tako da bova dobila pot $au_1u_2b$ iz roba ab in pot $cv_1v_2d$ iz roba cd. Nato bova povezala $u_1$ z $v_1$ in $u_2$ z $v_2$.
\end{document}